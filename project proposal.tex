\documentclass{article}
\usepackage[utf8]{inputenc}
\usepackage[english]{babel}

\usepackage[left = 4cm, right = 4cm]{geometry}
\usepackage{multicol}

\begin{document}

\begin{titlepage}
    \begin{center}
        \vspace*{1cm}

        \Huge
        \textbf{Creation of a Poker AI}

        \vspace{0.5cm}
        \LARGE
        Introduction to Artificial Intelligence Project Proposal

        \vspace{1.5cm}

        \textbf{Alexandros Apostolou, Tom Collins, Joseph Dowling, Arnold Gomes}

        \vfill


        \vspace{0.8cm}
        \Large
        University of Bristol
        \textbf{https://github.com/EMAT31530/ai-group-project-game-ai-team}
        15/01/2021

    \end{center}
\end{titlepage}

\begin{multicols*}{2}
\section{Introduction to the Problem}
We decided to base our project on an application relating to gaming, and develop an AI that is capable of playing games vs humans and improving itself. Some initial ideas we proposed were creating a Settlers of Catan AI or a Chess AI. We settled on creating a poker AI based on the Texas Hold'em variant of poker.
\newline

We chose to implement an AI in poker as we thought a table top card game would be simple to code and we would be able to focus on the AI side of it more. We were also interested in the topic of reinforcement learning and think that it will feature heavily in the creation of our AI.

Our objective with the AI is to start off with an incredibly basic AI that would randomly take actions with equal likelihood and then improve after playing many games versus other players. We are unsure at the moment whether we should let the AI learn from any player that it plays, experienced players only or a perfect AI. We may further explore these different possibilities as the AI may learn different behavioural patterns based on the subjects it plays.


\section{Potential Project Ideas}
Before we settled on poker, we briefly discussed the potential of creating an AI in other games. One of the possibilities we considered was a Settlers of Catan AI. We thought that it may be an area where AI hadn't been explored very deeply so there would be room for us developing something completely unique.
\newline

While it is a popular game, we decided against a Settler's of Catan AI after some initial testing on how to implement the game itself in Python. We found that it was a complex task to even program a working game due to the hexagonal structure of the board and the large variety of possible actions. It would be more difficult to find the time to focus on the AI aspect as the majority of the time would be spent creating a playable prototype of the game. It would also be very difficult to compare
\newline

We also discussed creating a Chess AI as it is another board game that was very common and seemingly more simple to program than Settler's of Catan. There is also lots of potential for AI to be developed and the best AI today are unbeatable by human opponents. However, we thought that due to the popularity of Chess, many AI developments in the field have already been explored so there would be nothing new that we could innovate.
\newline

We settled on poker because we thought it was simple to implement and even though there are lots of poker AI that have already been created, we believe that we can put our own unique spin on it. We are interested in exploring the area of bluffing in poker. This is a very human element of poker that is unique to each player and the tendencies they have when they bluff are most likely drawn from their inbuilt characteristics.

\section{Approach to AI}
As previously mentioned, we have decided that Reinforcement Learning will be our main AI methodology used in the training algorithm. The reason for this is that reinforcement learning is employed when the rewards for specific actions are delayed. The AI will take actions based on the state of the constantly changing environment, and then receive a reward or penalty at the end of the round. It will then learn what actions are best suited to take based on maximising the future reward. In Texas Hold'em, there are several possible changes to the environment to be considered, such as when a new community card is drawn, changes to players' bets (which would be indicative of their behaviour which can be based off the strength of their hand) and changes to the number of players depending on players that may fold.
\newline

We also need to create a way to calculate a reward for each action and the general disposition of the AI will be to maximise reward/minimise penalty. However, poker is an imperfect information game; not knowing the opponents cards adds a lot of complexity in the calculations to determine a reward for each action. In simple 2 player games with perfect information, it is possible to use game theory to determine an optimal strategy, representing a zero sum game. For poker, it will require the AI to learn what moves are likely to lead to rewards and what moves are likely to lead to penalty. The way that it will learn these is through simulating several games and analysing its decisions, determining what are the current optimal decisions and then re-evaluating again after several more trials.
\newline

Reinforcement learning is a good model to use here because it is able to learn from the outcome of its own decisions and decide on the best possible action based on the information it currently has. If it makes a mistake, it is highly unlikely that the same mistake will occur again as the model will correct itself.
\newline

A downside of reinforcement learning is that it is a very long process and requires a lot of computation and large amounts of data to achieve results. However, this downside is partially mitigated in our project as we are programming a game. This allows for large amounts of data to be gathered very easily as we can simulate the AI playing many games versus opponents we can program. A potential candidate for an opponent can also be the AI playing against itself with slight alterations in strategy. If both AI are trying to maximise their rewards, they will both learn from each other and after several iterations, both will evolve. They can then be evaluated separately and the one that performs worse can be discarded. This technique is similar to the use of a genetic algorithm and we will potentially explore this area further in the future.


\section{Literature Review}

Designing an AI to play a game and eventually beat its human experts has long been a great milestone in AI. These achievements have often been made in games such as chess and Go where both players have full information about the game and AI can practically be developed using brute force alone. Poker has received a lot of attention lately as the quintessential game of imperfect information, where just two cards kept secret per player present a host of challenges for developers. To draw inspiration for our project, we have examined some of the most prominent poker AI that have been developed in recent years.
\newline

One such AI is DeepStack \cite{Deepstack}, an AI designed to play heads up (two player) no-limit hold’em. The specific choice of game is motivated by the fact that Texas Hold’em is the most popular form of poker, and that there already existed many AI to play limit hold’em (where the amount a player can bet in any round is capped) since this greatly restricts the number of actions the AI can take and therefore substantially reduces computation. In approaching this frontier, the team designed an AI that doesn’t calculate an entire strategy before the game but rather approaches each situation in the game with a strategy calculated there and then. The key element is what the team calls ‘continual resolving’. DeepStack performs a heuristic search to a given depth limit – usually four – and uses approximations of the game to determine the value of each leaf node. The breadth of the search is similarly restricted by limiting the potential actions of the AI to folding, calling, 2 or 3 bet actions and all-in raises.
\newline

The computer works backwards at each decision point to build the strategy it took to get to that point; it then uses the heuristic search to determine the next move, abandoning its previous strategy. The advantage of this resolving is that the AI would otherwise have to search to the end of the hand to determine its strategy and this is practically infeasible. To calculate the values at the leaf nodes, the AI uses deep learning. In this, the AI has to determine its range (the probability distribution of hands they hold, given their actions) and the opponent’s counterfactuals (the value of potential hands they hold). This range is in turn developed in assuming that the opponent is playing an optimal Nash equilibrium strategy.
\newline

The soundness of this online strategy means that the AI has very low exploitability – essentially, the strategy has no weaknesses. It is therefore unsurprising that DeepStack consistently beat some of the best professional poker players over many hands of play.
\newline

Poker games with more than two players present a much greater challenge since no known polynomial time algorithm exists to compute the Nash equilibrium strategy, which is the theoretical unbeatable strategy in a two-player context. The team developing Pluribus \cite{Pluribus}, designed to play six-player no-limit hold’em, therefore decided to focus on empirical performance rather than theoretical soundness. Like DeepStack, Pluribus determines its strategy using self-play – rather than being fed data from human games, for example – but also uses a technique called abstraction. The AI attempts to solve a close approximation of the game that vastly reduces computation required: action abstraction lumps together similar actions – treating a bet of £400 and £401 as the exact same, for example – and information abstraction lumps together similar hands – treating a fourth highest card of a seven the same as a six, for example – in situations where the distinction is not important.
\newline

The self-play works through a technique called Monte Carlo counterfactual regret minimisation (MCCFR) to play hands against copies itself; after each hand, one of the copies evaluates its strategy and makes an adjustment to their strategy based on their performance. This process begins with the AI playing a random strategy and that initial poor strategy can end up having a lasting effect on their strategy for many iterations, so later strategies are given a higher weighting in the early stages of this process. With this self-play, the AI creates a blueprint strategy that it improves upon during the game.
\newline

Using depth limited search trees, the AI searches for better actions than those currently in its blueprint strategy. Like DeepStack, the AI needs to place a value on the leaf nodes based on an estimate of the value of that position if the opponents played a Nash equilibrium strategy from theron. The twist here though is that the AI also simulates the opponents playing strategies slightly biased from the optimal one; these strategies have weaknesses but also punish opponents for having weaknesses, so Pluribus is funneled towards a balanced strategy. Indeed, Pluribus also aims to reduce its exploitability by simulating the actions it would take in a situation with every possible hand, not just the one it is holding, to make sure that its strategy is correctly balanced. And so, like DeepStack, Pluribus triumphed over its human opponents during testing.
\newline

One of the most recent poker AI to have been developed is Rebel \cite{Rebel1} \cite{Rebel2}, which is a general algorithm for playing two-player zero-sum games. Rebel uses a combination of reinforcement learning and search to find its strategy. The key innovation, however, is that Rebel maintains a public belief state (PBS) for each player, which is the AI’s assessment of what other players think their range is. This allows the game to be treated as if there is perfect information, but just probabilistically. (One assumption here is that other players’ strategies are known, which is less applicable to playing against human opponents.) Like other poker AI, Rebel uses CFR to search for its strategy, and is proven to converge to a Nash equilibrium strategy in two-player zero-sum games. Rebel is one step towards an all-purpose AI that performs well in a variety of tasks, not just one.




\section{Initial Experimentation}
While we have not started developing our AI in depth, we have created several basic AI with predetermined strategies as a baseline. Examples of the strategies we have created are:

\begin{itemize}
    \item An AI that will always go all in on every hand, regardless of any other factors
    \item An AI that will fold on every hand
    \item An AI that will always check/call every hand
    \item An AI that will randomly select between call/checking, folding and raising by a random amount with equal probability
    \item A more sophisticated strategy that will calculate the strength of its hand and play accordingly
\end{itemize}

The simplistic nature of first four strategies will mean that they are only useful in testing the strength of our AI initially. Although the final strategy is more sophisticated, it is still deterministic. And due its simplicity, the AI should quickly learn how it plays and what is the optimal method to win.
\newline

We will require a way to benchmark our AI once it exhibits characteristics exceeds the performance of the simplistic strategies. There are several ways that we can do this and they are also related to the way we have discussed to train the AI. We can either train it versus other humans or versus a perfect poker AI that already exists.
\newline

Training the AI vs other humans would give us a large number of possible strategies that can be studied and learnt from. The AI would determine certain elements from each player's strategy that would be beneficial to have and incorporate them to create a better strategy. If the AI is playing very high skilled players, it would perform poorly but would most likely learn the best possible strategies a lot faster than if it plays lower skilled players. A possible downside of training against other humans is that if the level of the players is very low, the growth of the AI would be stunted. This is because if the AI can only learn from low level strategies, then it can at most only learn to beat low level strategies. When pitched against a higher skilled opponent, it would struggle to compete. This does not mean that the optimal solution is to only have our AI play against high skill opponents straight away. Although we have not explored this area yet, it may be better to play against opponents of gradually increasing skill level, which would lead to gradual development.
\newline

Another issue with training against other humans is the time it would take. Training an AI requires a large data set and gathering this amount of data from human players may require multiple thousands of trials which would take an incredibly long time to gather. This is a reason that it may be better to train our AI versus an already created Perfect AI as it would be easy to run many trials to train our AI. As stated before, it may be best to learn directly from an already perfect strategy rather than gradually building skill from various strategies it is facing.

We will test the effectiveness of our training algorithm by simulating many games of our AI against the designated opponent and storing the actions taken and outcomes that follow. This data is then used via the reinforcement learning to 'train' our AI. As time goes on, taking a graph of the AI's win rate over time there would ideally be a general increase, demonstrating the improvements of the AI thanks to our reinforcement learning. If we vary our training AI or method we can also quantify their efficacy by the rate in improvement of our AI's win rate over time.





\section{TB1 Topics and their relevancy}
%not sure whether to write a line or two about each topic and it's relevancy. main topics relevant are game theory and reinforcement learning
\begin{itemize}
    \item Linear regression - we don't think supervised learning in general is likely to be very applicable to the project. Although we may be able to generate or gather labelled data (the outcome of the hand, based on the strategy used), there are too many features involved in making a decision due to the recursive nature of the problem that supervised learning techniques would become unwieldy. One potential application we considered might be to develop a strategy and then try and represent it as a linear combination of different features e.g. bluff percentage on turn, fold percentage on river etc. but this would likely come later in the project after a strategy is developed.
    \item Linear classification - similar to above. We don't think this technique has the complexity to be applicable to the project.
    \item Unsupervised learning - we can't think of an application of the k-means or k nearest neighbour algorithms to our project.
    \item Decision trees - this is an area we are going to explore in more depth to evaluate whether it will be useful for our project or not. The AI is of course making a series of decisions but, like linear regression, we wonder if the features are too complex to be distilled into a decision tree, especially given the dynamic nature of the game.
    \item Search I - many poker AI use heuristic search procedures to find an optimal (or near optimal) action in each setting. Like existing poker AI, we would want to limit the breadth of the search (limiting the AI's actions at each node) as well as the depth (only looking so far into the future). Determining the value of the goal nodes would likely require some other techniques or offline learning to approximate the AI's situation.
    \item Markov Decision processes and Reinforcement learning - we think this will be perhaps the most important area of research for our project. It is clear that poker is nicely adapted to a framework of states, actions, rewards and transition probabilities. Combined with search algorithms, we believe that reinforcement learning will be the main way that the AI determines its actions on each move. The situation is complicated by the fact that the AI is playing against an opponent and we will therefore be considering some game theory (see below) but reinforcement learning algorithms such as Q-learning should be very useful nonetheless.
    \item Game Theory - playing against opponents of course lends itself to game theory. Though poker has imperfect information, it is still possible to find a Nash equilibrium strategy with two players and we think mini-max algorithms combined with the above techniques will allow our AI to search for actions effectively by assuming that its opponent plays an optimal strategy. Playing with more than two players is much harder and as such we probably won't consider this unless we find a practical workaround to simplify the problem.
    \item Ethics - we are likely to generate our own data through self play and as such there aren't any major data protection considerations. The team developing Pluribus decided not to open source their code since they were worried that it would encourage cheating on online poker sites; if we design an AI so powerful that it threatens the very existence of online poker itself then we will cross that bridge when we come to it but this is not currently on our radar.
\end{itemize}

\section{Future Work}
For the next teaching block, we have the following objectives:
\begin{itemize}
    \item To develop a learning algorithm that employs the concept of Reinforcement Learning, and perhaps some form of heuristic searches based upon our literature review, to train our AI.
    \item To develop rigorous testing algorithms and devise a way to statistically analyse the results so we can quantify the performance of our AI, allowing us to test the quality of our learning algorithm.
    \item Implement a GUI for presentation and ease of use when a human user is playing against the AIs.
\end{itemize}





\end{multicols*}









\begin{thebibliography}{9}
\bibitem{Deepstack}
Matej Moravcik, Martin Schmid, Neil Burch, Viliam Lisy, Dustin Morrill, Nolan Bard, Trevor Davis, Kevin Waugh, Michael Johanson, Michael Bowling
\textit{DeepStack: Expert-Level Artificial Intelligence in
Heads-Up No-Limit Poker}.
\texttt{https://arxiv.org/pdf/1701.01724.pdf}
Department of Computing Science, University of Alberta, Edmonton, Canada, 2017.

\bibitem{Pluribus}
Noam Brown, Tuomas Sandholm
\textit{Superhuman AI for multiplayer poker}
[\texttt{https://science.sciencemag.org/content/sci/365/6456/885.full.pdf}].
Published online, 2019.

\bibitem{Rebel1}
Noam Brown, Anton Bakhtin
\textit{ReBeL: A general game-playing AI bot that excels at poker and more}
\\\texttt{https://ai.facebook.com/blog/rebel-a-general-game-playing-ai-bot
-that-excels-at-poker-and-more/}
Published online, 2020

\bibitem{Rebel2}
Noam Brown, Anton Bakhtin, Adam Lerer, Qucheng Gong
\textit{Combining Deep Reinforcement Learning and Search
for Imperfect-Information Games}
\\\texttt{https://arxiv.org/pdf/2007.13544.pdf?fbclid=IwAR0kkrj60aCyXK
x-DUKLvtihYK1KCcnFS8ic6wr-ZEA4wv0AuqE29RJRyAk}
Published online, 2020





\end{thebibliography}




\end{document}
